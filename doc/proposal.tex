\documentclass{article}
\usepackage{authblk}
\usepackage{hyperref}
\usepackage{libertine}
\begin{document}

\title{Computers, Sound, and Music: Project Proposal}
\author[]{Yuxiang Jiang}
\affil[]{\textit {yux@pdx.edu}}
\maketitle

\section{Proposal Overview}

We propose to build a music synthesis engine. More specifically, the input of
our engine will be sheet music or other kind of digital sheet music format,
and the output will be audio synthesized by sample based
synthesis. In another word, we are kind of building a MIDI (or other format)
playback engine.

We would use free instrument samples from the internet. We found
\href{http://theremin.music.uiowa.edu/MIS.html}
  {The University of Iowa Musical Instrument Samples}%
, which
contains samples of all 88 keys of a piano, and various string
instrument samples.

For the input format, we have two choices:
\begin{itemize}
  \item We can use an existing digital format (MIDI or MusicXML)
so that we can reuse existing libraries to read them.
\item We can also design our own format and then convert existing
sheet music to it. This requires a lot more work but requires us to
learn more music notations.
\end{itemize}

\section{Challenges}

We know almost nothing about music theory and sample based synthesis. On
the other hand, we think that most of the existing digital sheet music formats
assume that their users should have an understanding of music theory.
So before we are able to implement our engine, we will have to learn these
prerequisites.

The other challenge lies in testing. How can we make sure that our engine
correctly reproduce the sound that the sheet music specifies? We know that
different musicians can have their own interpretation of music sheets so
directly comparing our result with music recording from human musicians
might not be the best way for testing. We should also find some reference
implementations and compare our results with theirs.

Yet another challenge is to support as many instruments as
possible. We expect that the synthesis parameters for different instruments
differ, so each instrument will take some time to implement. We will at
least support piano. And if time permits, we would also try to add
some string instruments.

\section{Milestones}

We would expect the project to be split into 4 steps:
\begin{itemize}
  \item Choose a digital sheet music format
  \item Implement a sample based synthesis engine
  \item Wire up the engine with the format we chose
  \item Compare our results with reference implementations
\end{itemize}

\section{Group}

I'd like to do this on my own. That is, my `group' consists of myself only.

\end{document}
